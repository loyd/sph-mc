\Introduction

Задача моделирования поведения жидкости находит много применений в различных отраслях. В физике, технике и математике основные требования предъявляются к физической корректности и точности моделирования, а не к визуальному результату, причём основная задача состоит именно в минимизации погрешности результата. В комьютерных играх и кинематографе наоборот, основополагающим является визуальный результат, а физическая корректность может быть частично нарушена.

Моделирование гидродинамики применяется во многих отраслях:
\begin{itemize}
  \item самолетостроение, ракетостроение, автомобилестроение (характеристики кузова, работа двигателя, сопла);
  \item промышленная химия (разделение веществ, химические реакторы);
  \item метеорология, геология (потоки жидкости сквозь пористые среды, песчаники, дамбы);
  \item медицина (потоки крови, лимфы);
  \item кинематограф;
  \item моделирование физики для интерактивного обучения.
\end{itemize}

Процесс моделирования требует большого количества вычислительных ресурсов, поэтому долгое время симуляция в реальном времени была невозможна. Однако быстрое развитие компьютерной техники, в особенности графических процессоров и сопровождающего их ПО, позволяет приспособить и оптимизировать комплекс методов для выполнения этой задачи.

Целью выполнения курсового проекта является реализация физического симулятора поведения жидкости, её взаимодействия с объектами. Симулятор должен быть способным работать в режиме реального времени.

Для достижения поставленных целей ставились следующие задачи:
\begin{enumerate}
  \item сбор материалов для реализации задачи;
  \item анализ алгоритмов решения поставленной задачи;
  \item разработка собственного алгоритма решения;
  \item проведение экспериментов, стресс-тестов для разработанного решения.
\end{enumerate}
